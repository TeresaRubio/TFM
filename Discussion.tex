El grueso de este proyecto se desglosa en tres bloques principales, que coinciden con los objetivos marcados desde el principio. La primera parte del trabajo, se centra en reproducir la identificaci�n taxon�mica de los microorganismos presentes en las muestras. En la segunda parte, se encuentra que los datos siguen la ley de Taylor, lo que permite explorar la estabilidad temporal de la microbiota en diferentes condiciones para entender la relaci�n con el estado de salud de los sujetos. Por �ltimo, se hace un estudio de las correlaciones entre microorganismos y se abren las puertas al uso de alternativas para medir interacciones, que es un campo donde a�n queda mucho por explorar debido a su complejidad.

El primer objetivo incluye la clasificaci�n taxon�mica de diversos genomas en una misma muestra. La idea original de reproducir el proceso llevado a cabo por los autores de los datos originales, se cumple satisfactoriamente. A pesar de emplear una versi�n m�s reciente tanto del software como de la base de datos, los resultados son similares (se comprueba por comparaci�n). Esto sirve para testar la reproducibilidad de los m�todos empleados que en determinadas ocasiones es importante. Por otro lado, el nivel taxon�mico alcanzado no es el m�s deseable. Se prefiere poder caracterizar los microorganismos a nivel de especie, o incluso de cepa, para poder dar un significado biol�gico a lo que ocurre en la microbiota. A nivel de g�nero es muy dif�cil especificar una sola funci�n, porque se trata de un grupo de especies que pueden englobar tanto organismos beneficiosos como pat�genos. La aproximaci�n mediante 16S es poco precisa y para un estudio m�s profundo es necesario el uso de la secuenciaci�n del genoma completo (\textit{shotgun}). Esta �ltima permite identificar las diferencias entre microorganismos aislados y conocer la informaci�n funcional y gen�tica, llegando incluso a detectar mutaciones en el genoma. Permite mostrar el contenido de genes de la comunidad, lo que es muy �til para definir las capacidades de la comunidad comparando con bases de datos para conocer el las funciones de esos genes.

Respecto al segundo objetivo, abarca todo el an�lisis de series temporales y se obtienen varios resultados interesantes. Caracterizar el comportamiento global del sistema no es trivial, pero gracias a la herramienta complexCruncher se puede extraer bastante informaci�n. Esta es la parte m�s matem�tica del proyecto, pero intenta explicar la biolog�a subyacente. Principalmente se consiguen distintos enfoques de la variabilidad del microbioma, lo que se relaciona con la salud del hospedador. Otra idea un poco diferente planteada en esta parte es identificar modos v�a descomposici�n de Fourier. Esto sirve para encontrar posibles periodicidades en el microbioma, es decir, comprobar si se repite alg�n patr�n a lo largo del tiempo (semanas, meses...). Los resultados no se a�aden a la memoria porque no se encuentran modos elementales en esta serie temporal. Quiz�s porque son necesarios m�s puntos temporales o quiz�s porque simplemente no existen estos patrones.

Por �ltimo, el tercer objetivo es inferir la din�mica del sistema. Se logran medir los comportamientos paralelos entre grupos, aunque no se ha llegado a definir cada una de las interacciones que se dan entre microorganismos. Conseguir una red coloreada por bloques de especies que interaccionan entre si ser� el siguiente objetivo a cumplir y ser�a ideal que adem�s se pudiera visualizar a trav�s del tiempo. Como es un campo en auge, se abren as� las puertas a futuros trabajos.

Como conclusiones finales al estudio, remarcar tres espec�ficas:
\begin{itemize}
\item
\end{itemize}

